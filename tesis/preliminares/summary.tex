%*******************************************************
% Summary
%*******************************************************

\newpage



\chapter*{Summary}
\addcontentsline{toc}{chapter}{Summary}
\section*{Brief Summary}



\section*{Resumen Extendido}

Durante la redacción de este trabajo estudiaremos un campo multidisciplinar que involucra por lo general tanto consideraciones propias de los intereses de las matemáticas como de los intereses del campo de la ingeniería informática. Por lo tanto, es difícil distinguir a grandes rasgos que partes son de mayor interés para un lector que pertenezca a solo una de las disciplinas. Sin embargo intentaremos diferenciar, en la medida de lo posible, que partes son de interés para cada una de dichas disciplinas. En caso de que esto no sea posible, justificaremos el interés que cada ciencia tiene.\\

La primera parte introduce el problema y estudia su complejidad. Procedemos a explicar que realizamos en cada capitulo.\\

\textbf{Capítulo 1:} En este capitulo exponemos los fundamentos de la lógica. El objetivo es introducir con la máxima formalidad posible el marco de trabajo durante todo el proyecto. 
Empezamos definimos el Álgebra Booleana como un retículo distributivo. Posteriormente, demostramos el teorema de Knaster-Tarski para retículos completos. Definimos a continuación la Lógica Proposicional como un Lenguaje Formal. De este modo primero especificamos la sintaxis y la semántica de este 'idioma'. Definimos una semántica basada en asignaciones y asignaciones parciales. 
Consideramos que este capítulo suscita un interés principal en el campo de la matemática. Sin embargo es recomendado que todo lector lo lea para poder comprender en toda su extensión los capítulos subsecuentes.\\

\textbf{Capítulo 2:} En este capítulo definimos formalmente lo que es un problema, e introducimos el problema SAT como un problema de decisión sobre el lenguaje de la lógica proposicional. Definimos las variaciones de este problema que suscitan más interés, y que estudiaremos en el siguiente problema. Para acabar el capítulo demostramos como se realiza un certificado de ejecución de SAT. Para ello demostramos la completitud del la regla de resolución. Por último hablamos sobre los problemas de satisfacción de restricciones.\\





\endinput
